\documentclass[10pt,a4paper]{article}
\usepackage[utf8]{inputenc}
\usepackage[english]{babel}
\usepackage{url}
\usepackage{lipsum}
\usepackage[textwidth=7in, total={170mm,257mm}]{geometry}
\author{\textit{Jørgen Ellingsen}}

\title{\textsc{Plagiarsism and cheating: two rotten eggs in the research egg basket}}

\begin{document}
\maketitle
\date{}
\bibliographystyle{vancouver}
% Introduction
Today's students will be the researchers, teachers, politicians and industry leaders of tomorrow. A recent study find a positive trend in students acceptance of academic dishonesty in most areas \cite{Molnar2015}, but nevertheless still a significant problem in higher education. 

Plagiarism is usually divided into intentional and unintentional plagiarism, and range from unintentional unreferenced citations to contracted cheating. A exploratory study of cheating by economics and business undergraduate students reveal that academtic dishonesty differ dramaticly in different parts of the world. Students admitting to cheat ranged from 5\% in Scandinavian countries to 88\% in Eastern European countries \cite{Teixeira2010}.
\\ \\ % Methods
This literature study examens pervious work to find possible reasoning and countermeasurs for academic dishonesty. 
\\ \\ % Results
Competition among students appear to be a positive thing, but getting ahead of competitors for jobs and academic positions is part of the incentive for intentional plagiarism and cheating. The majority of students consider plagiarism a minor problem or no problem at all \cite{Teixeira2010}.

A study on factors for plagiarism suggest that a large number of assignments and poor time management by students might lead them to take the easy way out and copy someone else's work. The same study also suggest that instructors who fail to motivate the students as a likley cause \cite{Comas2010}.

Students contracting out their work is hard to detect, and might be considered more fraudulent than other forms of plagiarism \cite{Walker2012}. 
\\ \\ % Discussion
Finding plagiarism can already be done by using software to search for similar sentences and figures in academic papers. This will likley only become easier with time, as large search engines like Google Scholar is indexing and refining their service. Contract cheating on the other hand, is very hard to detect. The work itself is original, and done by someone with knowledge and skill in academic writing. 

A focus on detection methods will help to reveal plagiarism, but for contract cheating this will most likley only encourage the cheaters to find ways to bypass the detection tecniques \cite{Walker2012}. At the same time, this will take the education staff's time and energy - time and energy that might be best used for inspiering and motivating students, giving them a good reason for why this is important to learn. This, combined with lectures on what plagiarism is, and why it is a serious academic offence, could hopefully lead to students making the right choice themselfs, instead of making the right choice out of fear of getting caught.

A lecturer who is perceived by the student as not interested in the subject he/she is teaching, giving the students theoretical assignments they are not motivated to do would create "ideal" conditions for plagiarism and cheating \cite{Comas2010}. Combined with short deadlines and/or high workload in other subjects, students are pushing the assignments to the last minute and are more likley to go for the easy solution \cite{Comas2010}. 

\bibliography{b}
\clearpage

\subsubsection*{Main Subject}
Academic honesty

\subsubsection*{Choose between these three subtopics}

- Open Access: Challenges and possibilities\\
- The five pillars of ethical research: honesty, trust, justice, respect and responsibility\\
- Plagarsism and cheating: two rotten eggs in the research egg basket
\subsubsection*{Max length}
Two pages including bibliography (~1000 words)
\subsubsection*{Use scholarly sources and cite them correctly}
ACM, IEEE Xplore, Science Direct, Oria and SpringerLink(link.springer.com), search engines like Google Scholar and ISI Web of science\\
Use the checklist and select documents to read - find the essence of the paper
\subsubsection*{Checklist}
- Title, is it relevant? \\
- What kind of document is it? \\
- Is it peer-reviewed? \\
- When is it published? is it still relevant?\\
- IMRAD Structure (Introduction, Methods, Results and Discussion)\\
- Does it have a methods chapter?\\
- Does it have a bibliography? \\
\subsubsection*{Due date}
Noon, October 21st 2016

\end{document}