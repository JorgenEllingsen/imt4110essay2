\documentclass[12pt,a4paper]{article}
\usepackage[utf8]{inputenc}
\usepackage[english]{babel}
\usepackage{url}
\usepackage{lipsum}
\usepackage[textwidth=7in, total={170mm,257mm}]{geometry}
\author{\textit{Jørgen Ellingsen}}

\title{\textsc{Plagiarism and cheating: two rotten eggs in the research egg basket}}

\begin{document}
\maketitle
\date{}
\bibliographystyle{vancouver}
% Introduction
Today's students will be the researchers, teachers, politicians and industry leaders of tomorrow. A recent study find a declining trend in students acceptance of academic dishonesty in most areas, but copying others' written work, among a few other, is revealed having a negative trend - that is, a higher acceptance of this type of plagiarism \cite{Molnar2015}. Another paper state that academic dishonesty have become widespread and pervasive in higher education \cite{Dalal2015}, although not empirical quantified research, it's worth noting that plagiarism and cheating is still a problem in higher education. New students will enter academia on a yearly basis unaware of the pitfalls and consequences both for themselves and the university they represent. 
%Plagiarism, copying someone else's work and passing it of as your own, can be divided into intentional and unintentional plagiarism. Intentional plagiarism is when you know that you're using someone else's work, and you know that it should have been cited and referenced. Unintentional plagiarism is when a person lack the knowledge of how to cite and reference others work that your work uses and rely on. 
%The former is usually considered cheating, because you know you're doing something you're not supposed to - thus trying to cheat the system.
\\ \\ % Methods
This literature study examines relevant papers to find possible reasoning and countermeasures for academic dishonesty. 
\\ \\ % Results
Competition among students appear to be a positive thing, but getting ahead of competitors for jobs and academic positions is part of the incentive for intentional plagiarism and cheating \cite{Teixeira2010}. The  majority of students also consider plagiarism a minor problem, or no problem at all \cite{Teixeira2010}.

A study on factors for plagiarism suggest that a large number of assignments and poor time management by students might lead them to take the easy way out and copy someone else's work. The same study also suggest that instructors who fail to motivate the students is a likely cause for plagiarism \cite{Comas2010}.

A article on causes for student cheating suggests that witnessing others' cheat, or think that cheating occurring regularly on their institution, reinforce the chances of students cheating themselves \cite{Rettinger2009}. The same article also states that giving the students a sense that thay can be caught, and that the consequences can be severe, is important to counteract cheating \cite{Rettinger2009}.

It also seems the institutional approach is based on policies, procedures and punitive or disciplinary sanctions, rather than reflection, dialogue and inner understanding \cite{Dalal2015}.

Another problem is that many students perceive plagiarism as copying large blocks of text from other people's work without any form of acknowledgement, and the majority of students and professors have no problem identifying it as dishonest behaviour \cite{Childers2016}. This kind of plagiarism is easy to understand, and therefor easy to attribute to intentional misconduct \cite{Childers2016}, but plagiarism is alot more complex than that.

As an example, less then a third of the participants in a study on complex citation issues identified the reuse of ideas alone as plagiarism \cite{Childers2016}.
\\ \\ % Discussion
Finding plagiarism can already be done by using software to search for similar sentences and figures in academic papers. This will likely only become easier with time, as large search engines like Google Scholar is indexing and refining their service. Contract cheating on the other hand, is very hard to detect. The work itself is original, and done by someone with knowledge and skill in academic writing. 

A focus on detection methods will help to reveal plagiarism, but for contract cheating this will most likely only encourage the cheaters to find ways to bypass the detection techniques \cite{Walker2012}. At the same time, this will take the education staff's time and energy - time and energy that might be best used for inspiring and motivating students, and increase the education of students on why they need the skill, and not just the formal qualifications \cite{Walker2012}. This, combined with lectures on what plagiarism is, and why it is a serious academic offense, could hopefully lead to students making the right choice themselves, instead of making the right choice out of fear of getting caught. 

Its hard to determine whether behavioral change is based off the fear of detection and punishment, or a change in the students academic moral \cite{Dalal2015}. Students might therefor cheat as soon as they belive the chances of getting caught is less likely \cite{Dalal2015}, suggesting that propper education in academic honesty might be a better solution for the academic community as a whole. The results of the paper on complex citation issues also indicate that more emphasis might be needed on copying of direct work, but also the source of ideas \cite{Childers2016}. This is a pitfall for unknowledgable students, and might make them unintentionally commit plagiarism.

A lecturer who is perceived by the student as not interested in the subject he/she is teaching and giving the students theoretical assignments they are not motivated to do would create "ideal" conditions for plagiarism and cheating \cite{Comas2010}. Short deadlines and/or high workload in other subjects, increases the likelyhood that students are pushing the assignments to the last minute and are more likely to go for the easy solution \cite{Comas2010}. This is hard to do anything about in a study with several courses with different instructors, but worth noting.
%A exploratory study of cheating by economics and business undergraduate students reveal that academic dishonesty differ dramatically in different parts of the world. Students admitting to cheat ranged from 5\% in Scandinavian countries to 88\% in Eastern European countries \cite{Teixeira2010}.
\\ \\
\bibliography{b}
\clearpage

\subsubsection*{Main Subject}
Academic honesty

\subsubsection*{Choose between these three subtopics}

- Open Access: Challenges and possibilities\\
- The five pillars of ethical research: honesty, trust, justice, respect and responsibility\\
- Plagarsism and cheating: two rotten eggs in the research egg basket
\subsubsection*{Max length}
Two pages including bibliography (~1000 words)
\subsubsection*{Use scholarly sources and cite them correctly}
ACM, IEEE Xplore, Science Direct, Oria and SpringerLink(link.springer.com), search engines like Google Scholar and ISI Web of science\\
Use the checklist and select documents to read - find the essence of the paper
\subsubsection*{Checklist}
- Title, is it relevant? \\
- What kind of document is it? \\
- Is it peer-reviewed? \\
- When is it published? is it still relevant?\\
- IMRAD Structure (Introduction, Methods, Results and Discussion)\\
- Does it have a methods chapter?\\
- Does it have a bibliography? \\
\subsubsection*{Due date}
Noon, October 21st 2016

\end{document}